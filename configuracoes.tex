\documentclass[
	12pt,
	openright,
	oneside,
	a4paper,
	sumario=tradicional,
	english,
	french,
	spanish,
	brazil,
	]{abntex2}

\usepackage{lmodern}
\usepackage[T1]{fontenc}
\usepackage[utf8]{inputenc}
\usepackage{indentfirst}
\usepackage{color}
\usepackage{graphicx}
\usepackage{microtype}
\usepackage{array}
\usepackage{lipsum}
\usepackage[brazilian,hyperpageref]{backref}
\usepackage[alf,abnt-emphasize=bf]{abntex2cite}
\usepackage{amsmath}
\usepackage{amssymb}
\usepackage{lastpage}
\usepackage{listings}
\usepackage{pdflscape}
\usepackage{dirtree}
\usepackage{ulem}
\renewcommand\DTstyle{\rmfamily}
\usepackage{longtable}
\usepackage{multirow}
\usepackage{comment}
\usepackage{pdfpages}
\usepackage{booktabs}

%package para quebra de linha em url
\usepackage{xurl}

% Package para código fonte
\usepackage{minted}  

% Package para controle de elementos flutuantes, diciona H aos parametros de piosicionamento
\usepackage{float} 


%Packages para criação de pseudo códigos
\usepackage{algpseudocode}
\usepackage{algorithm2e}
\RestyleAlgo{ruled}
\DontPrintSemicolon

%comandos customizados para este projeto

\newcommand*\rot{\rotatebox{90}}
\renewcommand{\backrefpagesname}{Citado na(s) página(s):~}
\renewcommand{\backref}{}
\renewcommand*{\backrefalt}[4]{
	\ifcase #1 
		Nenhuma citação no texto.%
	\or
		Citado na página #2.%
	\else
		Citado #1 vezes nas páginas #2.%
	\fi}%

\newcommand{\commentBlock}[1]{} %comenta um bloco de texto



% ---
% Informações de dados para CAPA e FOLHA DE ROSTO 
% ---
\titulo{DETECÇÃO DE DIMENSÕES DE BAGAGENS AEROPORTUÁRIAS A PARTIR DE DISPOSITIVOS DE BAIXO CUSTO}
\autor{Vitor de Almeida Silva \\
Prof. Orientador: Prof. Dr. Marcos Paulino Roriz Junior \\
Profa. Coorientadora: Profa. Dra. Michelle Carvalho Galvão da Silva Pinto Bandeira}
\local{Aparecida de Goiânia - GO}
\data{Dezembro, 2023}
\instituicao{%
  Universidade Federal de Goiás -- UFG
  \par
  Faculdade de Ciências e Tecnologia
  \par
  Programa de Pós-Graduação em Engenharia de Produção}
\tipotrabalho{Texto de dissertação (Mestrado)}
\preambulo{Texto de dissertação submetido para defesa no Programa de Pós-Graduação em Engenharia de Produção.}
\definecolor{blue}{RGB}{41,5,195}
\makeatletter
\hypersetup{
		pdftitle={\@title}, 
		pdfauthor={\@author},
    	pdfsubject={\imprimirpreambulo},
	    pdfcreator={LaTeX with abnTeX2},
		pdfkeywords={Demand Forecasting}{Combination Forecasting}{Combination Optimization}{Industrial Application}, 
		colorlinks=true,
    	linkcolor=blue,
    	citecolor=blue,
    	filecolor=magenta,
		urlcolor=blue,
		bookmarksdepth=4
}
\makeatother
\setlength{\parindent}{1.3cm}
\setlength{\parskip}{0.2cm}
\makeindex




% ----------------------------------------------------
% CORREÇÃO DO SUMÁRIO
% ----------------------------------------------------
% \usepackage{etoc}

%\etocsetstyle{chapter} 
%{}
%{\addvspace{.5ex}\setlength{\leftskip}{1.5cm}\noindent}
%{\llap{\makebox[1.5cm][l]{\bfseries\etocnumber}}\etocname
%  \hspace{10pt}\nobreak\dotfill\hspace{10pt}\etocpage\par}
%{}
%\etocsetstyle{section}
%{}
%{\addvspace{.5ex}\noindent\setlength{\leftskip}{1.5cm}\noindent}
%{\llap{\makebox[1.5cm][l]{{\etocnumber}}}\etocname
%  \hspace{10pt}\nobreak\dotfill\hspace{10pt}{\etocpage}\par}
%{}
%\etocsetstyle{subsection}
%{}
%{\addvspace{.5ex}\noindent\setlength{\leftskip}{1.5cm}\noindent}
%{\llap{\makebox[1.5cm][l]{{\bfseries\etocnumber}}}\etocname
%  \hspace{10pt}\nobreak\dotfill\hspace{10pt}{\etocpage}\par}
%{}
%\etocsettocstyle{\chapter*{\normalfont\bfseries SUMÁRIO}}{}
%\setcounter{tocdepth}{1}
%\setcounter{secnumdepth}{2}

%----------------------------------------------------
% CORREÇÃO DAS SECTIONS E SUBSECTIONS NO TEXTO
%----------------------------------------------------
% Finalmente, corrigi negrito nos numeros em chap e sect.
% \renewcommand{\chapnumfont}{\normalfont\bfseries} %ok
% \renewcommand{\chaptitlefont}{\bfseries\normalfont}


% \let\oldsubsection=\subsection
% Save the existing sectioning commands

% \renewcommand{\subsection}[1]{{%
%   \setsecnumformat{{\bfseries\thesubsection\quad}}%
%   \oldsubsection{#1}}} 