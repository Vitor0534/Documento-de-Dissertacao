\imprimircapa
\imprimirfolhaderosto

\begin{fichacatalografica}
    \vspace*{15cm} 
    \hrule 
    \begin{center} 
    \begin{minipage}[c]{12.5cm} 
        \imprimirautor
        \hspace{0.5cm} \imprimirtitulo / \imprimirautor. --
        \imprimirlocal, \imprimirdata-
        \hspace{0.5cm} \pageref{LastPage} p. : il.(alguma color.); 30 cm.\\
        \hspace{0.5cm} \imprimirorientadorRotulo \imprimirorientador\\
        \hspace{0.5cm}
        \parbox[t]{\textwidth}{\imprimirtipotrabalho~--~\imprimirinstituicao,
        \imprimirdata.}\\
        \hspace{0.5cm}
        1. Reconstrução 3D. 
        2. \textit{Self Bag Drop}.
        3. Dimensões de Bagagens.
        4. visão computacional.
        I. Prof. Dr. Marcos Paulino Rozis Junior.
        II. Profa. Dra. Michelle Carvalho Galvão da Silva Pinto Bandeira 
        III. Universidade Federal de Goiás.
        IV. Programa de Pós-graduação em Engenharia de Produção
        V. Detecção de dimensões de bagagens aeroportuárias a partir de dispositivos de baixo custo\\
        \hspace{8.75cm} CDU xx:xxx:xxx.x\\
    \end{minipage}
    \end{center}
    \hrule
\end{fichacatalografica}

\begin{folhadeaprovacao}
    \begin{center}
        {\ABNTEXchapterfont\large\imprimirautor}
        \vspace*{\fill}\vspace*{\fill}
        
        {\ABNTEXchapterfont\bfseries\Large\imprimirtitulo}
        \vspace*{\fill}
        \hspace{.45\textwidth}
        \begin{minipage}{.5\textwidth}
            \vspace{0.8cm}
            \imprimirpreambulo
        \end{minipage}%
        \vspace*{\fill}
    \end{center}
    
    \imprimirlocal, 20 de novembro de 2023:
    \assinatura{\textbf{Prof. Dr. Marcos Paulino Rozis Junior} \\ Orientador -- Universidade Federal de Goiás}
    \assinatura{\textbf{Profa. Dra. Michelle Carvalho Galvão da Silva Pinto Bandeira} \\ Coorientadora -- Universidade Federal de Goiás}
    \assinatura{\textbf{Prof. Dr. Diogo de Souza Rabelo} \\ Membro Interno -- Universidade Federal de Goiás}
    \assinatura{\textbf{Prof. Dr. Li Weigang} \\ Membro Externo -- Universidade de Brasília}
    
    \begin{center}
        \vspace*{0.5cm}
        {\large\imprimirlocal}
        \par
        {\large\imprimirdata}
        \vspace*{1cm}
    \end{center}
\end{folhadeaprovacao}


\begin{comment}
        \begin{dedicatoria}
            \vspace*{\fill}
            Este trabalho é dedicado...
            \vspace*{\fill}
        \end{dedicatoria}
        
        \begin{agradecimentos}
            Os agradecimentos...
        \end{agradecimentos}
\end{comment}






\begin{epigrafe}
    \vspace*{\fill}
    \begin{flushright}
        \textit{‘‘Não importa o quanto tente, \\ você sozinho não pode mudar o mundo.\\  Mas este é o lado bonito do mundo.‘‘\\
        (L - Death Note)}
    \end{flushright}
\end{epigrafe}



\begin{resumo}
Estudos recentes mostram um aumento significativo no tempo de embarque de passageiros, variando de 22 minutos para 40 minutos no período de 1990 a 2009. Uma das causas desse atraso é o processo de check-in, em maior parte devido à verificação das dimensões de bagagens de mão e de porão. As dimensões influenciam no armazenamento das bagagens e no gasto com a carga, caso não estejam nos padrões o cliente poderá pagar taxas extras e ser orientado a voltar à fila de \textit{check-in}. 
Para amenizar o problema, empresas estão investindo em \textit{self bag drop}. Nessa lógica, o passageiro fica responsável pelas medidas. Entretanto, pode-se gerar erros devido ao formato da bagagem e a complexidade de lidar com objetos despachados em conjunto. O presente trabalho constatou que alguns dispositivos conseguem obter automaticamente as dimensões de objetos/bagagens, em especial equipamentos baseados na tecnologia a LASER. Contudo, são equipamentos com preços expressivos e que comumente requerem uma instrumentação e modificação do ambiente para sua instalação. Com isso, esta pesquisa busca investigar a viabilidade técnica de se utilizar dispositivos de baixo custo na obtenção das dimensões de bagagens aeroportuárias.
Dentre as técnicas de baixo custo, destaca-se o uso do sensor de profundidade Microsoft Kinect, capaz de obter uma nuvem de pontos (\textit{point cloud}) do objeto em análise. Com base neste sensor, desenvolveu-se um algoritmo para capturar, montar e analisar a \textit{point cloud} gerada, obtendo assim a dimensão da bagagem. 
Para validar a abordagem, foi construído um protótipo que contém uma esteira e uma estrutura para fixar o sensor, permitindo a configuração de velocidade e parâmetros de captura dos dados, como passo de amostragem e região de captura. 
Os testes realizados indicam ser possível obter dados de profundidade, largura e altura com precisão pelo uso do sensor de profundidade Microsoft Kinect V2, com uma menor exatidão para a altura. Tais resultados demonstram o potencial de uso dessa alternativa de baixo custo no gerenciamento das operações de embarque e investimentos das companhias aéreas.


    \vspace{\onelineskip}
    \noindent
    \textbf{Palavras-chaves}: Reconstrução 3D. \textit{self bag drop}. dimensões de bagagens. visão computacional.
\end{resumo}


\begin{comment}
    \begin{resumo}[Summary]
        \begin{otherlanguage*}{English}
            This is a summary in English...
            \vspace{\onelineskip}
            \noindent
            \textbf{Keywords}: latex. abntex. publication of texts.
        \end{otherlanguage*}
    \end{resumo}
\end{comment}


\pdfbookmark[0]{\listfigurename}{lof}
\listoffigures*
\cleardoublepage

\pdfbookmark[0]{\listtablename}{lot}
\listoftables*
\cleardoublepage


\begin{SingleSpace}
\begin{siglas}
    \item[2D ou $R^2$] 2 dimensões
    \item[3D ou $R^3$] 3 dimensões
    \item[ANAC] Agência Nacional de Aviação Civil
    \item[AABB] Axes Align Bounding Box
    \item[DBSCAN] Density-Based Spatial Clustering of Applications with Noise
    \item[KPI] Key Performance Indicators
    \item[LASER] Light Amplification by the Stimulated Emission of Radiation
    \item[LiDAR] Light Detection and Ranging
    \item[MAE] Mean Absolute Error
    \item[MBB] Minimum Bounding Box
    \item[OMBB] Oriented Minimum Bounding Box
    \item[PDI] Processamento Digital de Imagens
    \item[RGB] Red, Green, and Blue
  \end{siglas}
\end{SingleSpace}
\pdfbookmark[0]{\contentsname}{toc}
\tableofcontents*
\cleardoublepage

\textual